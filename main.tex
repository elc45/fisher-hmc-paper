\documentclass{scrartcl}

\usepackage{amsmath}
\usepackage{amssymb}
\usepackage{amsthm}
\usepackage{mathtools}

%\usepackage[urw-garamond]{mathdesign}
%\usepackage{notomath}
%\usepackage{newpxtext,newpxmath}
%\usepackage{garamondx}
%\usepackage[garamondx,cmbraces]{newtxmath}
%\usepackage{libertinus}

\usepackage[math-style=ISO, bold-style=ISO]{unicode-math}
%\usepackage[math-style=TeX, bold-style=TeX]{unicode-math}

\setmainfont{EB Garamond}
\setmathfont{Garamond-Math.otf}[StylisticSet={2,7,9}]
\addtokomafont{disposition}{\rmfamily}


%\usepackage[T1]{fontenc}
%\usepackage{mlmodern}
%\usepackage{lmodern}
\usepackage[english]{babel}
%\usepackage[nocolor]{hyperref}
%\usepackage{lualatex-math}
\usepackage{hyperref}

\usepackage{xcolor}
\hypersetup{
    colorlinks,
    linkcolor={red!50!black},
    citecolor={blue!50!black},
    urlcolor={blue!80!black}
}

\DeclareMathOperator{\eigvals}{eigvals}
\DeclareMathOperator{\diag}{diag}
\DeclareMathOperator{\CovStr}{Cov}
\DeclareMathOperator{\VarStr}{Var}
\DeclareMathOperator{\EStr}{E}
\DeclareMathOperator*{\argmax}{arg\,max}
\DeclareMathOperator*{\argmin}{arg\,min}
\DeclareMathOperator{\vecSpan}{span}

\DeclarePairedDelimiter{\norm}{\lVert}{\rVert}
%\DeclarePairedDelimiter{\Var}{\VarStr[}{]}
%\DeclarePairedDelimiter{\Cov}{\CovStr}{]}
\DeclarePairedDelimiterXPP\Var[1]{\VarStr}(){}{#1}
\DeclarePairedDelimiterXPP\Cov[1]{\CovStr}(){}{#1}
\DeclarePairedDelimiterXPP\E[1]{\EStr}(){}{#1}

\newtheorem{thm}{Theorem}
\newtheorem{cor}{Corollary}

\title{Fisher Divergence for mass matrix adaptation}
\author{Adrian Seyboldt}


\begin{document}

\maketitle

\section{Introduction}

Mass matrix is critical for sampler performance...

Adapt mass matrix during tuning based on first samples...

In the case of normal posterior $N(\mu, \Sigma)$ we use the condition number of
$\Sigma$ as an estimate of how difficult it would be to sample from this
posterior with an identity mass matrix.

\[
\kappa(\Sigma) = \frac{\min{\eigvals(\Sigma)}}{\max{\eigvals(\Sigma)}}
\]

If we use a mass matrix $\hat{\Sigma}^{-1}$, we can compute the
condition number seen by the sampler as

\[
\kappa(\Sigma, \hat{\Sigma})
  = \frac{\min{\eigvals(\Sigma, \hat\Sigma)}}{\max{\eigvals(\Sigma, \hat\Sigma)}},
\]

where $\eigvals(\Sigma, \hat\Sigma)$ refers to the generalized eigenvalues of
$\Sigma$ with respect to $\hat\Sigma$.

A better estimate can be found here,
\url{https://arxiv.org/pdf/1905.09813.pdf}, but for our purposes the simpler
condition number will do fine.

Currently, most implementations (at least Stan + PyMC) use a regularized
version of the empirical covariance or its diagonal as inverse mass matrix.
(mention normalizing flows etc. based on ADVI?)

%In the following we use
%\[
%  \kappa_1(\Sigma) = d(\Sigma, I) = \norm{\log(\eigvals(\Sigma))}.
%\]


%Choosing a different mass matrix is equivalent to sampling in the coordinate
%system defined by the scaled eigenvectors of the inverse mass matrix, so
%for an inverse mass matrix $\hat{\Sigma}$ we find

%\[
%  \kappa_1(\hat{\Sigma}| \Sigma)
%    = d(\hat{\Sigma}, \Sigma)
%    = \lVert \log(\eigvals(\hat{\Sigma}, \Sigma) \rVert,
%\]

%where $\eigvals(\hat{\Sigma}, \Sigma)$ are the generalized eigenvalues of
%$\hat{\Sigma}$ with respect to $\Sigma$.

\subsection{Use gradients of posterior log density}

This means however, that we do not use an additional source of information that
we have available anyway: the gradients of the posterior log density at the
positions of the draws.

Since the gradients transform covariantly, where the draws transform
contravariantly, the covariance of gradients is the inverse covariance matrix
for a Gaussian posterior. This indicates that the covariance of the gradients
gives us an estimate for $\Sigma^{-1}$, while the variance of the draws
themselves gives an estimate for $\Sigma$. If we have few draws compared to the
dimensionality of the problem, the empirical covariances therefore contain
information about the large and small eigenvalues of the true covariance of the
posterior respectably. But since both of those are important for a good mass
matrix, it stands to reason that we can use the gradient information to improve
mass matrix estimates.

Estimates of the variance with only the draws are also limited by the
Cramér-Rao bound, which is no longer the case if we consider additional
information. For instance, for a 1D Gaussian posterior one draw with its
gradient is already enough to uniquely identify the posterior distribution, and
as we will see it also provides a lot of information about the posterior in the
higher dimensional case.

\subsection{Fisher Divergence}

Choosing a mass matrix is equivalent to fitting a normal distribution to the
posterior, where the inverse covariance matrix of the fitted distribution is
the mass matrix.

A natural way to incorporate the gradient information into this covariance
estimation is by minimizing the fisher divergence (see e.g.
\url{https://arxiv.org/pdf/1905.05284.pdf}) between our estimated normal
distribution with density $q$ the actual posterior distribution with density
$p$:

\[
  F(p, q)
    = \int \lVert \nabla \log p(\theta)
      - \nabla \log q(\theta)\rVert^2 p(\theta)d\theta
\]

It is not immediately obvious which norm we should use here. A natural choice
becomes apparent if we remember that the mass matrix as used in HMC itself
defines a norm on the tangent parameter space. We propose to use this norm,
which means that we are looking for a distribution with density $q$ who's
gradients are close to the gradients of the target distribution $p$ \emph{as
seen by the sampler}:

$$
  F(p, q)
    = \int \lVert \nabla \log p(\theta)
      - \nabla \log q(\theta)\rVert_{\hat\Sigma}^2 p(\theta)d\theta
$$

Where the norm on the tangent space is given by $\lVert x\lVert_{\hat\Sigma}^2
= x^T \hat\Sigma x$, where $\hat\Sigma$ is the covariance of our fitted
distribution $q$.

This is equivalent to finding an affine transformation $\phi(x) = Ax + \mu$
such that

\[
F(\phi)
  = \int \lVert \nabla \log p_\phi(\theta)
    - \nabla \log N(x \mid 0, 1) \rVert^2 p(\theta) d\theta
\]
is minimal, where $p_\phi(\theta) = \lvert J_\phi\rvert^{1/2} p(\phi(\theta))$
is the transformed density and $AA^T = \hat\Sigma$ corresponds to the
covariance of our fitted Gaussian distribution, or the inverse mass matrix.

This definition is now valid for arbitrary families of diffeomorphisms $\phi$.

During sampling we have $N$ draws $\theta_i$ and gradients $\nabla \log
p(\theta_i)$, so we approximate $F$ by its Monte Carlo estimate:

\[
\hat{F}(p, q)
  = \frac{1}{N} \sum_i \lVert \nabla p(\theta_i)
    - \nabla q(\theta_i)\rVert^2_\Sigma
\]

or
\[
\hat{F}(\phi)
  = \frac{1}{N} \sum_i
    \norm{\nabla \log p_\phi(\theta_i) - \nabla \log N(\theta_i \mid 0, 1)}
\]

\subsection{Properties of minimization problem}

\begin{thm}
If $\phi_{\mu, A} = Ax + \mu$ and $X$ with density $p$ is such that $\Cov{X}$
and $\Cov{\nabla \log(p(X))}$ exist and are positive definite, then
$F(\phi_{\mu, A})$ is minimal iff $\mu = \E{X}$ and $AA^T$ is the
geodesic mean of $\Cov{X}$ and $\Cov{\nabla \log(p(X))}^{-1}$.

Put differently, $AA^T$ is such that
\[
d(\Cov{X}, AA^T) + d(AA^T, \Cov{\nabla\log(p(X))}^{-1}),
\]
is minimal, where $d(\Sigma, \Omega) = \norm{\log(\eigvals(\Sigma, \Omega))}$
is the geodesic distance of symmetric positive definite matrices
$\Sigma$ and $\Omega$ with respect to the intrinsic metric.

$\hat{F}$ is minimal for all $A$ with $\hat{\Sigma} AA^T\hat{\Sigma} =
\hat{\Omega}$, where $\hat{\Sigma} = \Cov{X_i}$ is the empirical
covariance of the draws and $\hat{\Omega} = \Cov{\nabla \log(p(X_i))}$ is
the empirical covariance of the gradients.
\end{thm}

\begin{proof}
See appendix. (TODO)
\end{proof}

So if we interpret both $\Cov{X}$ and $\Cov{\nabla \log(p(X))}^{-1}$ as
possible choices for the inverse mass matrix, minimizing $F$ leads to the mean
of those two choices. If we have more dimensions than draws, this minimum (or
mean) is not unique however, so we will need regularization of some kind in
order to find a unique mass matrix estimate.

\begin{cor}
If $\phi$ is as above and additionally $X ~ \sim N(\mu', \Sigma')$,
then the minimum of $F$ is at $\mu = \mu'$ and $\Sigma = \Sigma'$, because in
this case $\Cov{X} = \Cov{\nabla \log(p)}^{-1}$.

Let $k$ be the number of draws in $n$ dimensions and let $\hat A$ be the linear
part of the minimizer of $\hat F$. Then there exists a $k$-dimensional
subspace of $\vecSpan(X_i, \nabla\log p(X_i))$ in which $\hat A\hat A^T$
matches the true covariance $\Sigma$ exactly.
\end{cor}

This means if our posterior is a Gaussian, we get better estimates for the mass
matrix the more draws we have, and if we have more draws than dimensions our
mass matrix estimate will be perfect.

\begin{cor}
Given a space of transformation $\phi_{\mu, \sigma}(X) = \mu + \sigma \odot X$
(this corresponds to diagonal mass matrices) $F$ is minimal if $\mu =
E_p[X]$ and $\sigma^2 = \sqrt{\frac{\Var{X}}{\Var{\nabla \log(p(X))}}}$
and

\[
\sigma
  = \argmin_{\sigma} d(\Cov{X}, \diag(\sigma)^2)
    + d(\diag(\sigma)^2, \Cov{\nabla\log(p(X))}^{-1}),
\]

$\hat F$ is minimal if
\[
\sigma^2 = \sqrt{\frac{\Var{X_i}}{\Var{\nabla \log(p(X_i)}}}.
\]
\end{cor}

This last result motivates a simple modification of the status quo diagonal
mass matrix adaptation. Instead of adapting based on the marginal variance of
each parameter, we adapt based on the geometric mean of the marginal variance
and inverse marginal variance of the gradients.

In the next part of this paper we will investigate this modification in more
detail, and compare an implementation of it in nutpie with the current default
choice.

In later parts we will show how we can develop the ideas further for
sub-quadratic mass matrix adaptation that still can deal with highly correlated
posteriors.

The final part will investigate further generalizations to non-linear
transformations using neural networks.

\iffalse
\section{Unsorted}

\begin{proof}
Assuming $q = N(\hat{\mu}, \hat{\Sigma})$ and differentiating by $\hat{\mu}$ and
a symmetric $\hat{\Sigma}$ gives us

\begin{align*}
\hat{\mu} &= \hat{\Sigma}\overline{\nabla \log p(\theta_i)} + \overline{\theta_i}
C &= \hat{\Sigma}P\hat{\Sigma} \\
\end{align*}

where
\begin{align*}
P &= \Cov{\nabla \log p(\theta_i), \theta_i} \\
C &= \Cov{\theta_i, \theta_i} \\
\overline{y_i} &= \frac{1}{N}\sum_i y_i
\end{align*}
\end{proof}

The Lyapunov equation for $\hat{\Sigma}$ has a symmetric positive definite solution if
$C$ is positive definite and $P$ has only unique negative eigenvalues (ref?).
Those conditions are satisfied when $p$ is a normal distribution and $N >
\dim(\theta)$. *In this case $\hat{\Sigma}$ is exactly $\Sigma$, even if
the draws $x_i$ are not from the posterior distribution*. (I don't have a
proper proof yet, works out numerically though, and I think it shouldn't be too
hard to do? (last words\ldots))

So this completely solves the case where the posterior distribution is exactly Gaussian,
and we have more draws than dimensions.

\section{Some closed form solutions}

\subsection{Diagonal covariance}

If we restrict $q$ to a normal distribution with diagonal covariance, we find

$$
\hat{\Sigma}_{ii} = \sqrt{\frac{C_{ii}}{P_{ii}}}.
$$

If $p$ happens to also be a normal distribution with diagonal covariance, this
estimate will be exact as soon as we have more than one draw.

\subsection{Full covariance}
\subsubsection{Number of draws larger than posterior dimension}

The solution of (?) is the true covariance.

\subsubsection{Number of draws smaller than posterior dimension}

Here we run into issues when solving the Lyapunov equation, because $P$ and $C$ both
have eigenvalues that are zero.

I don't have a good theoretical framework for dealing with this, but after some
experimentation I think I found a solution that works reasonably well in
practice (it outperformed all other ideas I tried for all covariance matrices I
tried (see notebook)).

We first estimate a diagonal matrix
\[
\diag(D) = \sqrt{\frac{\hat{\Var{\theta_i}}}{\hat{\Var{\nabla \log p(\theta_i)}}}}.
\]

The intuition behind this is that large posterior variance implies large
variance of the samples, but also small variance of the gradients. (By the way,
using this as a diagonal inverse mass matrix seems to outperform the diagonal
mass matrix based only on the samples in most cases, even without further
modifications that follow).

We then compute $P'$ and $C'$ based on scaled versions of gradients and draws
($\theta' = \theta D^{1/2}, \nabla \log p' = \nabla \log p D^{-1/2}$).
We then use an eigenvalue decomposition and SVD respectably to set eigenvalues
that are $\leq 0$ in the case of $C'$ and eigenvalues that are $\geq 0$ in the case
of $P'$ to a small positive (for $C'$) or negative (for $P'$) number $\epsilon$.

This preserves the property of fitting a normal distribution exactly when we have
more draws than dimensions, but allows us to still get good (but maybe not ideal)
estimates if we have fewer samples.

\subsubsection{More expensive but better(?) solution}

Write mass matrix as $D(U(\Sigma - I)U^T + I)D$ and use natural gradient decent to
minimize fisher divergence.

\subsection{Non-Gaussian posterior}

I don't know of any method to evaluate different choices for the mass matrix
that trying a bunch of models with each and comparing the ESS per gradient
evaluation.

In some experiments that I did this new mass matrix or the $D$ defined above
performed similar or better for most posteriors.

\subsection{Computational cost}

We have to consider the cost of finding the mass matrix, which in a windowed tuning
method we have to do only a few times, and the cost of leapfrog and momentum resampling.

The cost to compute $\hat{\Sigma}$ is dominated by solving the Lyapunov
equation, for which I currently use the Bartels-Stewart algorithm with $15n^3$
flops as implemented in scipy.

We then have to do matrix vector multiplications with $O(n^2)$ operations in
each leapfrog step. If $n$ is large we can reduce this by further approximating
$\hat{\Sigma} \approx D^{1/2}(U\hat{\Sigma}'U^T + (I - UU^T))D^{1/2}$, where $U\in
\mathbb{R}^{n\times k}$ are the largest and smallest eigenvectors of the
correlation matrix. This reduces the cost to $O(kn)$.

It might also be possible to use Krylov methods to directly estimate $U$ and the
corresponding eigenvalues instead of solving the complete Lyapunov equation.

\fi

\end{document}
